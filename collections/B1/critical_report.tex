% created by make_collection.py
% manual adjustments:
% - page break before HerEy 78, 85, 86/43, 934
% - manual break in scoring of HerEy 107
% - \DeclareTOCStyleEntry (2x) with numwidth=3em
\documentclass[tocdir=../../tmp/B1]{ees}

\DeclareTOCStyleEntry[
  indent=0pt,
  beforeskip=0pt,
  entrynumberformat=\textcolor{eesred},
  numwidth=3em,
  linefill=\hfill,
  pagenumberbox=\pnumbox
]{tocline}{section}

\makeatletter
\DeclareTOCStyleEntry[
  indent=0pt,
  beforeskip=-\parskip,
  entrynumberformat=\@gobble,
  entryformat=\ltseries,
  numwidth=3em,
  linefill=\hfill,
  pagenumberbox=\pnumbox,
  pagenumberformat=\ltseries
]{tocline}{subsection}
\makeatother

\begin{document}

\title{[collection – add custom title page]}
\date{2024-06-28}
\license{cc-by-sa-4.0}
\def\MetadataLilypondVersion{2.24.3}
\def\MetadataEESToolsVersion{v2023.10.0}
\repository{edition-esser-skala/eybler-sacred-music}
\version{v2024.06.0}
\checksum{dba68e568b34e9ba17e59f6c8bb236980a7a6ea0}
\def\MetadataQRCode{\relax}
\eesTitlePage

\chapter{Critical Report}

In general, this edition closely follows the respective principal source.
Any changes that were introduced by the editor are indicated
by italic type (lyrics, dynamics and directives), parentheses
(expressive marks and bass figures) or dashes (slurs and ties).
Accidentals are used according to modern conventions.
For further details, consult the Editorial Guidelines
available on the Edition’s webpage.

\section{Abbreviations}

\begin{abbreviations}
\abbr{A}{alto}
\abbr{B}{bass}
\abbr{S}{soprano}
\abbr{T}{tenor}
\abbr{b}{basses}
\abbr{clno}{clarion}
\abbr{cl}{clarinet}
\abbr{cor}{horn}
\abbr{fag}{bassoon}
\abbr{fl}{flute}
\abbr{harm}{Harmonium}
\abbr{ob}{oboe}
\abbr{org}{organ}
\abbr{timp}{timpani}
\abbr{trb}{trombone}
\abbr{vla}{viola}
\abbr{vlc}{violoncello}
\abbr{vl}{violin}
\end{abbreviations}

\section{Per te Dei genitrix · HerEy 44}

\begin{xltabular}{\linewidth}{@{} >\itshape l X}
genre & gradual \\
festival & de Beatæ Mariæ Virginis \\
scoring & S, A, T, B (coro), 2 ob, 2 cl (C), 2 fag, 2 cor (C), 2 vl, vla, vlc, b, org \\
\end{xltabular}

\begin{sources}
  
\sourceitem%
  {A1}%
  {A-Ws}%
  {Cod. 571/16}%
  {autograph manuscript (principal source)}%
  {1828}%
  {}%
  {public domain}%
  {}%
  {full score; 11 pages}

\end{sources}


\textlt{\textit{Lyrics}\\Per te Dei genitrix nobis est data vita perdita,\\ quae de caelo suscepisti prolem\\ et mundo genuisti salvatorem.}

\section{Nocte surgentes · HerEy 47}

\begin{xltabular}{\linewidth}{@{} >\itshape l X}
genre & gradual \\
festival & de Tempore \\
scoring & S, A, T, B (coro), 2 ob, 2 cl (A), 2 fag, 2 clno (D), timp (D–A), 2 vl, vla, vlc, b, org \\
\end{xltabular}

\begin{sources}
  
\sourceitem%
  {A1}%
  {A-Ws}%
  {567 (3)}%
  {autograph manuscript (principal source)}%
  {1800-08}%
  {}%
  {public domain}%
  {}%
  {full score; 16 pages}


\sourceitem%
  {B1}%
  {A-Ed}%
  {B 278}%
  {manuscript copy}%
  {}%
  {600038212}%
  {public domain}%
  {https://dommusikarchiv.martinus.at/site/werkverzeichnis/gallery/120.html}%
  {17 parts (S (2×), A, T, B, ob 1, ob 2, clno 1, clno 2, timp, vl 1 (2×), vl 2 (2×), vla, vlne, org)}

\end{sources}

\begin{xltabular}{\linewidth}{ll X}
\toprule
\itshape Bar & \itshape Staff & \itshape Description \\
\midrule \endhead
– & – & fag 1/2 share a staff with timp and are written in a darker ink,
        possibly indicating that they have been added later.
        cl 1/2 are written on the same staff as clno 1/2,
        but in the same ink. \\

\bottomrule
\end{xltabular}

\textlt{\textit{Lyrics}\\Nocte surgentes vigilemus omnes,\\ semper in psalmis meditemur atque\\ viribus totis Domino canamus\\ dulciter hymnos.\\ (Liber Hymnarius 1983, p. 224)}

\section{Tua est potentia · HerEy 50}

\begin{xltabular}{\linewidth}{@{} >\itshape l X}
genre & gradual \\
festival & de tempore \\
scoring & S, A, T, B (coro), 2 cl (\flat B), 2 fag, 2 clno (\flat E), [2 trb], timp (\flat E–\flat B), 2 vl, vla, vlc, b, org \\
\end{xltabular}

\begin{sources}
  
\sourceitem%
  {A1}%
  {A-Ws}%
  {Cod. 571/9}%
  {autograph manuscript (principal source)}%
  {1822-01}%
  {}%
  {public domain}%
  {}%
  {full score; 11 pages}


\sourceitem%
  {C1}%
  {A-Wn}%
  {HK.2144}%
  {print}%
  {1826}%
  {991018078}%
  {public domain}%
  {https://data.onb.ac.at/rec/AC14328864}%
  {full score; Tobias Haslinger, Wien, plate number 4792}


\sourceitem%
  {C2}%
  {A-Wn}%
  {F4.Baden.90}%
  {print}%
  {1826}%
  {654000103}%
  {public domain}%
  {https://data.onb.ac.at/rec/AC14328853}%
  {17 parts (S, A, T, B, cl 1, cl 2, fag 1/2, clno 1 and 2 on one sheet, trb 1 and 2 on one sheet, timp, vl 1, vl 2, vla, vlc/b, org); Tobias Haslinger, Wien, plate number 4795}

\end{sources}

\begin{xltabular}{\linewidth}{ll X}
\toprule
\itshape Bar & \itshape Staff & \itshape Description \\
\midrule \endhead
–   & trb   & Trombones are only included in \C1 and \C2. \\
2ff & vl, vla, org & In bars 2, 3, 6, and 30, all sources show the rhythm
              \eighthNote–\semiquaverRestDotted–%
              tuplet of 3×\thirtysecondNote. Here, the
              \semiquaverRestDotted\ has been emended to a \semiquaverRest
              in order to yield a plausible rhythm. \\
25  & T     & The 3rd \quarterNote\ was originally \flat b4 in \A1,
              but has been corrected (likely by later hand) to f′4, since
              the original note would lead to parallel fifths with A.
              Nevertheless, \flat b4~still appears in \C1 and \C2. \\
28  & trb 1 & 2nd \halfNote\ in \C2: \halfNoteRest \\

\bottomrule
\end{xltabular}

\textlt{\textit{Lyrics}\\Tua est potentia, tuum regnum, Domine,\\ tu es super omnes gentes.\\ Da pacem, Domine, in diebus nostris.\\ (\bibleverse{IChr}(29:11))}

\section{Specie tua · HerEy 53}

\begin{xltabular}{\linewidth}{@{} >\itshape l X}
genre & gradual \\
festival & de quavis Sancta \\
scoring & S, A, T, B (coro), org \\
\end{xltabular}

\begin{sources}
  
\sourceitem%
  {A1}%
  {A-Ws}%
  {Cod. 571/1}%
  {autograph manuscript (principal source)}%
  {1796}%
  {}%
  {public domain}%
  {}%
  {full score; 6 pages}

\end{sources}


\textlt{\textit{Lyrics}\\Specie tua, et pulchritudine tua\\ intende, prospere procede, et regna.\\ Propter veritatem et mansuetudinem et iustitiam:\\ et deducet te mirabiliter dextera tua.\\ (\bibleverse{Ps}(45/44:5))}

\section{Christus factus est · HerEy 54}

\begin{xltabular}{\linewidth}{@{} >\itshape l X}
genre & gradual \\
festival & Coena Domini, Inventio et Exultatio Sanctae Crucis \\
scoring & S, A, T, B (coro), org \\
\end{xltabular}

\begin{sources}
  
\sourceitem%
  {A1}%
  {A-Ws}%
  {Cod. 571/2}%
  {autograph manuscript (principal source)}%
  {1797}%
  {}%
  {public domain}%
  {}%
  {full score; 4 pages}

\end{sources}


\textlt{\textit{Lyrics}\\Christus factus est pro nobis\\ obediens usque ad mortem,\\ mortem autem crucis.\\ Propter quod et Deus exaltavit illum\\ et dedit illi nomen,\\ quod est super omne nomen.\\ (\bibleverse{Phil}(2:8,9))}

\section{Non in multitudine · HerEy 56}

\begin{xltabular}{\linewidth}{@{} >\itshape l X}
genre & gradual \\
festival & de tempore \\
scoring & S, A, T, B (coro), 2 ob, 2 cl (C), 2 fag, 2 vl, vla, vlc, b, org \\
\end{xltabular}

\begin{sources}
  
\sourceitem%
  {A1}%
  {A-Ws}%
  {Cod. 571/11}%
  {autograph manuscript (principal source)}%
  {1823-08}%
  {}%
  {public domain}%
  {}%
  {full score; 10 pages}


\sourceitem%
  {C1}%
  {A-Wn}%
  {HK.2144}%
  {print}%
  {1831}%
  {991018071}%
  {public domain}%
  {https://data.onb.ac.at/rec/AC14328848}%
  {full score; Tobias Haslinger, Wien, plate number 5560}


\sourceitem%
  {C2}%
  {A-Wn}%
  {F4.Baden.88}%
  {print}%
  {1831}%
  {654000115}%
  {public domain}%
  {https://data.onb.ac.at/rec/AC14266063}%
  {15 parts (S, A, T, B, ob 1, ob 2, cl 1, cl 2, fag 1, fag 2, vl 1, vl 2, vla, vlne, org); Tobias Haslinger, Wien, plate number 5563}

\end{sources}

\begin{xltabular}{\linewidth}{ll X}
\toprule
\itshape Bar & \itshape Staff & \itshape Description \\
\midrule \endhead
43 & vla & 6th to 8th \eighthNote\ in \C2: 3 × e8 \\

\bottomrule
\end{xltabular}

\textlt{\textit{Lyrics}\\Non in multitudine est virtus tua, Domine.\\ Nec superbi ab initio placuerunt tibi:\\ sed humilium et mansuetorum semper placuit tibi deprecatio.\\ (\bibleverse{Jdt}(9:16))}

\section{Dies sanctificatus · HerEy 61}

\begin{xltabular}{\linewidth}{@{} >\itshape l X}
genre & gradual \\
festival & Nativitas Domini \\
scoring & S, A, T, B (coro), 2 cl (C), 2 fag, 2 cor (G), 2 vl, vla, vlc, b, org \\
\end{xltabular}

\begin{sources}
  
\sourceitem%
  {A1}%
  {A-Ws}%
  {Cod. 571/15}%
  {autograph manuscript (principal source)}%
  {1827}%
  {}%
  {public domain}%
  {}%
  {full score; 10 pages}


\sourceitem%
  {C1}%
  {A-Wn}%
  {N/A}%
  {print}%
  {1829}%
  {991018064}%
  {public domain}%
  {}%
  {full score; Tobias Haslinger, Wien, plate number 5244}


\sourceitem%
  {C2}%
  {A-Wn}%
  {F4.Baden.87}%
  {print}%
  {1829}%
  {654000106}%
  {public domain}%
  {https://data.onb.ac.at/rec/AC14265987}%
  {15 parts (S, A, T, B, cl 1, cl 2, fag 1, fag 2, cor 1, cor 2, vl 1, vl 2, vla, vlc/b, org); Tobias Haslinger, Wien, plate number 5247}

\end{sources}

\begin{xltabular}{\linewidth}{ll X}
\toprule
\itshape Bar & \itshape Staff & \itshape Description \\
\midrule \endhead
1 & vla & 1st \quarterNote\ in \C2: g8–g16–b16 \\

\bottomrule
\end{xltabular}

\textlt{\textit{Lyrics}\\Dies sanctificatus illuxit nobis,\\ venite gentes, et adorate Dominum,\\ quia hodie descendit super terram.}

\clearpage
\section{Tui sunt cœli · HerEy 78}

\begin{xltabular}{\linewidth}{@{} >\itshape l X}
genre & offertorium \\
festival & Nativitas Domini \\
scoring & S, A, T, B (coro), 2 cl (C), 2 fag, 2 cor (C), 2 clno (C), timp (C–G), 2 vl, vla, vlc, b, org \\
\end{xltabular}

\begin{sources}
  
\sourceitem%
  {A1}%
  {A-Ws}%
  {Cod. 735/14}%
  {autograph manuscript (principal source)}%
  {1827}%
  {}%
  {public domain}%
  {}%
  {full score; 16 pages}


\sourceitem%
  {C1}%
  {A-Wn}%
  {HK.2145}%
  {print}%
  {1829}%
  {991018079}%
  {public domain}%
  {https://data.onb.ac.at/rec/AC14328867}%
  {full score; Tobias Haslinger, Wien, plate number 5245}


\sourceitem%
  {C2}%
  {A-Wn}%
  {MS69264-4°/4}%
  {print}%
  {1829}%
  {654000118}%
  {public domain}%
  {https://data.onb.ac.at/rec/AC09306797}%
  {18 parts (S, A, T, B, cl 1, cl 2, fag 1, fag 2, cor 1, cor 2, clno 1, clno 2, timp, vl 1, vl 2, vla, vlc/b, org); Tobias Haslinger, Wien, plate number 5248}

\end{sources}

\begin{xltabular}{\linewidth}{ll X}
\toprule
\itshape Bar & \itshape Staff & \itshape Description \\
\midrule \endhead
45  & vl 2  & 3rd \quarterNote\ in \A1, \C1, and \C2: b′4 \\
102 & fag 1 & 1st \quarterNote\ in \C2: b4 \\
107 & fag 1 & 1st to 5th \eighthNote\ in \C2: \flat B2–\flat B′8 \\
120 & cl 2  & 3rd \quarterNote\ in \C2: c″4 \\

\bottomrule
\end{xltabular}

\textlt{\textit{Lyrics}\\Tui sunt coeli, et tua est terra:\\ Orbem terrarum, et plenitudinem eius tu fundasti:\\ Iustitia et iudicium praeparatio sedis tuae.\\ (\bibleverse{Ps}(89/88:12,15))}

\clearpage
\section{Terra tremuit · HerEy 85}

\begin{xltabular}{\linewidth}{@{} >\itshape l X}
genre & offertorium \\
festival & Resurrectio Domini \\
scoring & S, A, T, B (coro), 2 ob, 2 fag, 2 clno (C), timp (C–G), 2 vl, vla, vlc, b, org \\
\end{xltabular}

\begin{sources}
  
\sourceitem%
  {A1}%
  {A-Ws}%
  {571}%
  {autograph manuscript (principal source)}%
  {1797}%
  {}%
  {public domain}%
  {}%
  {full score; 16 pages}


\sourceitem%
  {B1}%
  {A-Wn}%
  {Mus.Hs. 733}%
  {manuscript copy}%
  {}%
  {600110969}%
  {public domain}%
  {https://data.onb.ac.at/rec/AC14266115}%
  {18 parts (S (2×), A (2×), T, B, ob 1, ob 2, fag 1, fag 2, clno 1, clno 2, timp, vl 1, vl 2, vla, vlne, org)}


\sourceitem%
  {C1}%
  {A-Wn}%
  {F24.St.Peter.E161(II)}%
  {print}%
  {1929}%
  {}%
  {public domain}%
  {https://data.onb.ac.at/rec/AC14328862}%
  {piano reduction (8 pages) and 17 parts (S, A , T, B, ob 1, ob 2, fag 1, fag 2, clno 1, clno 2, timp, vl 1, vl 2, vla, vlc, vlne, org); Anton Böhm \& Sohn, Augsburg–Wien, plate number 7109}

\end{sources}


\textlt{\textit{Lyrics}\\Terra tremuit, et quievit,\\ dum resurgeret Deus in iudicio.\\ (\bibleverse{Ps}(76/75:9,10))}

\clearpage
\section{Si consistant · Unam petii · HerEy 86/43}

\begin{xltabular}{\linewidth}{@{} >\itshape l X}
genre & offertorium \\
festival & De Tempore \\
scoring & 2 T, 2 B (solo), S, A, T, B (coro), 2 ob, 2 fag, 2 cor (C), 2 clno (C), timp (C–G), 2 vl, vla, vlc, b, harm, org \\
\end{xltabular}

\begin{sources}
  
\sourceitem%
  {A1}%
  {A-Ws}%
  {571}%
  {autograph manuscript (principal source)}%
  {1805}%
  {}%
  {public domain}%
  {}%
  {full score; 17 pages}


\sourceitem%
  {A2}%
  {A-Ws}%
  {569/3}%
  {autograph manuscript}%
  {no later than 1827}%
  {}%
  {public domain}%
  {}%
  {full score; 5 pages}


\sourceitem%
  {C1}%
  {A-Wn}%
  {HK.2145}%
  {print}%
  {1827}%
  {991018075}%
  {public domain}%
  {https://data.onb.ac.at/rec/AC14328856}%
  {full score; Tobias Haslinger, Wien, plate number 5013}


\sourceitem%
  {C2}%
  {A-Wn}%
  {F4.Baden.81}%
  {print}%
  {1827}%
  {654000116}%
  {public domain}%
  {https://data.onb.ac.at/rec/AC14266098}%
  {20 parts (S, A, T 1, T 2, B 1, B 2, ob 1, ob 2, fag 1, fag 2, cor 1, cor 2, clno 1, clno 2, timp, vl 1, vl 2, vla, vlc/b, org); Tobias Haslinger, Wien, plate number 5016}

\end{sources}

\begin{xltabular}{\linewidth}{ll X}
\toprule
\itshape Bar & \itshape Staff & \itshape Description \\
\midrule \endhead
–   & –      & The 1805 version of this work (\A1, HerEy 86)
               comprises the first section (“Si consistant”, 45 bars),
               a middle section (“Unam petii”, 69 bars) for harmonium
               (denoted “Baritono”) and male choir (TTBB),
               and the final section (“Si consistant”, 43 bars).
               By contrast, the 1827 versions (\A2 and prints) replace the
               harmonium in the middle section by 2 ob, 2 fag, and 2 cor.
               Version 1827a, represented by \A2 (HerEy 43), is similar
               to \A1: It lacks bars 71–75 and is therefore five bars
               shorter (i.e., 64 bars in total).
               Moreover, in bar 70, there are slight changes to the choir.
               By contrast, version 1827b, represented by \C1 and \C2,
               contains an even shorter middle section (31 bars in total),
               which lacks bars 77–109. Thus, when performing this version,
               one has to jump from the end of bar 76 to the beginning of
               the final section (as indicated by the segnos),
               and the male choir has to sing the 1st \quarterNote. \\
\midrule
–   & cor    & In the first and final section,
               cor only appear in \C1 and \C2. \\
9   & clno 1 & 1st \quarterNote\ in \C2: c′4 \\
28  & fag 2  & 3rd \quarterNote\ in \C2: \flat e4 \\
32  & T      & 4th \quarterNote\ in \C2 (only T 1): \flat e′4 \\
34  & cor 1  & 3rd \quarterNote\ in \C2: c″4 \\
46  & –      & tempo indication in \C2: “Adagio cantabile” \\
65  & T 1    & \C1 contains grace notes on the 9th (\sharp f″)
               and 11th (e″) \sixteenthNote. \\
115 & coro   & The 1st \quarterNote\ is only to be sung if version 1827b
               of the middle section is performed. \\
124–152 & –  & In \A1, these bars are indicated by \textit{vide} marks
               referring to bars 11–39 of the first section. \\

\bottomrule
\end{xltabular}

\textlt{\textit{Lyrics}\\Si consistant adversum me castra,\\ non timebit cor meum.\\ Si exurgat adversum me proelium,\\ in hoc ego sperabo.\\[1ex] Unam petii a Domino,\\ hanc requiram a Domino,\\ ut inhabitem in domo Domini\\ omnes dies vitae meae.\\ (\bibleverse{Ps}(27/26:3,4))}

\clearpage
\section{Fremit mare cum furore · HerEy 93}

\begin{xltabular}{\linewidth}{@{} >\itshape l X}
genre & offertorium \\
festival & De Tempore \\
scoring & S (solo), S, A, T, B (coro), 2 ob, cl solo (\flat B), 2 cl (\flat B), 2 fag, 2 clno (D), timp (D–A), 2 vl, vla, vlc solo, b, org \\
\end{xltabular}

\begin{sources}
  
\sourceitem%
  {A1}%
  {A-Ws}%
  {567 (4)}%
  {autograph manuscript (principal source)}%
  {1800-08}%
  {}%
  {public domain}%
  {}%
  {full score; 32 pages}


\sourceitem%
  {C1}%
  {A-Wn}%
  {HK.2525}%
  {print}%
  {1814}%
  {991018067}%
  {public domain}%
  {https://data.onb.ac.at/rec/AC14328843}%
  {10 parts (S solo, S, A, T, B, cl solo, vl 1, vl 2, vla, vlc/b); Stamperia chimica sul Graben (Chemische Druckerey), Wien, plate number 2137}


\sourceitem%
  {D1}%
  {D-NATk}%
  {NA/SP (E-22)}%
  {manuscript not used for this edition}%
  {1800-1830}%
  {455039871}%
  {public domain}%
  {https://mirador.acdh.oeaw.ac.at/musikarchivspitz/D-NATk_E22/}%
  {23 parts (S solo, S (2×), A (2×), T (2×), B (2×), ob 1, ob 2, cl, fag, clno 1, clno 2, b-trb, timp, vl 1(2×), vl 2, vla, vlc/b, org)}

\end{sources}

\begin{xltabular}{\linewidth}{ll X}
\toprule
\itshape Bar & \itshape Staff & \itshape Description \\
\midrule \endhead
–   & –    & Articulations and dynamics are exclusively taken from \A1,
             since they are highly inconsistent in \C1. \\
10  & vla  & 1st \halfNote\ in \C1: \sharp d′2 \\
55  & vl 2 & 6th/last \eighthNote\ in \C1: \flat b8 \\
82  & vl 1 & 1st \quarterNote\ in \C1: \flat e″+c′′′4 \\
83–167 & – & There are two versions of the middle section:
             The first version (which likely represents the earlier one)
             comprises all (85) bars and a solo for vlc. The second version
             (probably created in 1814 when Eybler revised the work
             for print \C1) omits bars 92–108 (here indicated by segnos)
             and contains a solo for cl. This is also the version
             reproduced in \C1 and was likely considered as the final version
             by Eybler (cf. his autograph catalogue of works,
             where this work is listed as offertorium no. 5:
             “in der Mitte mit Soprano und Clarinetto Solo”). \\
83  & vlc  & In \A1, a treble clef without transposition is used,
             so that vlc would sound one octave higher
             (i.e., starting with b″). \\
92  & org  & bar in \C1: \flat e2. \\
142, 152 & S & The upper voice is only written in \A1 (in small font.) \\
163 & cl   & last \sixteenthNote\ in \C1: a′16 \\
202–237 & cl, fag & These parts may have been added at a later timepoint,
             since their ink is different,
             and they are labeled with pencil. \\
214 & T    & 1st \halfNote\ in \C1: g2 \\
226 & vl 2 & 1st \eighthNote\ in \C1: \sharp f′8 \\
226 & vla  & 1st \eighthNote\ in \C1: d′8 \\
226 & A    & grace note added by editor \\

\bottomrule
\end{xltabular}

\textlt{\textit{Lyrics}\\Fremit mare cum furore\\ coelum undique obscuratur,\\ stridet fulmen cum terrore,\\ cor oppressum cruciatur.\\ Vivam adhuc sola spe.\\ Genus omne Deo creatum\\[1ex] nunc est triste, nunc beatum,\\ crescat ergo spes in me,\\ quid desperam? quid pavescam?}

\section{Reges Tharsis · HerEy 107}

\begin{xltabular}{\linewidth}{@{} >\itshape l X}
genre & offertorium \\
festival & Epiphanias \\
scoring & S, A, T, B (coro), fl, 2 ob, 2 fag, 2 cor (\flat B), 2 clno (\flat B), 3 trb, timp (\flat B–F),\newline 2 vl, vla, vlc, b, org \\
\end{xltabular}

\begin{sources}
  
\sourceitem%
  {A1}%
  {A-Ws}%
  {568 (3)}%
  {autograph manuscript (principal source)}%
  {1807}%
  {}%
  {public domain}%
  {}%
  {full score; 17 pages}


\sourceitem%
  {B1}%
  {A-Wn}%
  {F24.St.Peter.E151(II)}%
  {manuscript copy}%
  {1827}%
  {654000117}%
  {public domain}%
  {https://data.onb.ac.at/rec/AC14328853}%
  {four manuscript parts: fl (by Anton Diabelli), 3 trb (by Joseph Greipel)}


\sourceitem%
  {C1}%
  {A-Wn}%
  {HK.2145}%
  {print}%
  {1827}%
  {991018074}%
  {public domain}%
  {https://data.onb.ac.at/rec/AC14328852}%
  {full score; Tobias Haslinger, Wien, plate number 5047}


\sourceitem%
  {C2}%
  {A-Wn}%
  {F24.St.Peter.E151(II)}%
  {print}%
  {1827}%
  {654000117}%
  {public domain}%
  {https://data.onb.ac.at/rec/AC14328853}%
  {18 parts (S, A, T, B, ob 1, ob 2, fag 1, fag 2, cor 1, cor 2, clno 1, clno 2, timp, vl 1, vl 2, vla, vlc/b, org); Tobias Haslinger, Wien, plate number 5050}


\sourceitem%
  {E1}%
  {A-Wn}%
  {MS7845-4°/10}%
  {print not used for this edition}%
  {1928}%
  {}%
  {public domain}%
  {https://data.onb.ac.at/rec/AC09173367}%
  {conductor's score; Anton Böhm \& Sohn, Augsburg–Wien, plate number 6847}

\end{sources}

\begin{xltabular}{\linewidth}{ll X}
\toprule
\itshape Bar & \itshape Staff & \itshape Description \\
\midrule \endhead
–  & –     & Parts for fl and trb are later additions, only appearing in \B1.
             Parts for cor are included in \C1 and \C2 and thus have
             likely been sanctioned by Eybler. \\
24 & vla   & last \eighthNote\ in \C1: g′8 \\
38 & fag 1 & 1st \halfNote\ in \C1 and \C2: \flat e′2 \\
50 & fag 2 & 3rd \quarterNote\ in \C1 and \C2: \flat b4 \\
69 & trb 3 & 1st \quarterNote\ in \B1: f4 \\

\bottomrule
\end{xltabular}

\textlt{\textit{Lyrics}\\Reges Tharsis et insulae munera offerent,\\ reges Arabum et Saba dona adducent,\\ et adorabunt eum omnes reges terrae,\\ omnes gentes servient ei.\\ (\bibleverse{Ps}(72/71:10-11))}

\section{De profundis · HerEy 132}

\begin{xltabular}{\linewidth}{@{} >\itshape l X}
genre & psalm \\
festival & – \\
scoring & S, A, T, B (solo), S, A, T, B (coro), 2 ob, 2 cl (\flat B), 2 fag, 2 clno (\flat B), 3 trb, timp (\flat B–F), 2 vl, vla, vlc, b, org \\
\end{xltabular}

\begin{sources}
  
\sourceitem%
  {A1}%
  {A-Wn}%
  {16591}%
  {autograph manuscript (principal source)}%
  {1803}%
  {}%
  {public domain}%
  {}%
  {full score, 52 pages; bound as last work in one volume with the Requiem and Libera (HerEy 37)}


\sourceitem%
  {E1}%
  {A-Wn}%
  {HK.2147/1}%
  {print not used for this edition}%
  {1850}%
  {600243144}%
  {public domain}%
  {https://data.onb.ac.at/rec/AC14265982}%
  {full score, 31 pages; digitized version available at \url{https://data.onb.ac.at/rec/AC14266094}}

\end{sources}

\begin{xltabular}{\linewidth}{ll X}
\toprule
\itshape Bar & \itshape Staff & \itshape Description \\
\midrule \endhead
–   & –    & \E1 is a shortened version (comprising 199 bars in total):
             Bars 68 to 186 have been replaced with a short transition
             comprising three bars, and bars 239 to 249 and 284 to 363
             have been deleted. Moreover, the final fugue uses the
             lyrics “Alleluia”. \\
–   & trb  & These parts are indicated by the directive “3 Tromboni in Tutti
             con le Parti cantanti.”, written on the first page in red ink. \\
21  & –    & “con moto” has been added to the tempo indication with pencil. \\
195 & –    & Tempo indication has been added with pencil. \\

\bottomrule
\end{xltabular}

\textlt{\textit{Lyrics}\\De profundis clamavi ad te, Domine,\\ Domine, exaudi vocem meam.\\ Fiant aures tuae intendentes\\ in vocem deprecationis meae.\\ Si observaveris iniquitates Domine,\\ Domine, quis sustinebit?\\ Quia apud te propitiatio est,\\ et propter legem tuam sustinui te, Domine.\\ Sustinuit anima mea in verbo eius,\\ speravit anima mea in Domino.\\ A custodia matutina usque ad noctem\\ speret Israel in Domino,\\ Quia apud Dominum misericordia,\\ et copiosa redemptio apud eum,\\ et ipse redimet Israel\\ ex omnibus iniquitatibus eius.\\ Gloria Patri et Filio et Spiritui Sancto,\\ sicut erat nunc et semper\\ et in saecula saeculorum, amen.\\ (\bibleverse{Ps}(130/129:))}


\eesToc{}

\cleardoublepage%
\pagenumbering{arabic}%
\setcounter{page}{1}%
\includepdf[pages=-,link=true,linkname=score]{../../tmp/B1/full_score.pdf}%

\end{document}
