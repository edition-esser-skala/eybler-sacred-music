% created by make_collection.py
% manual adjustments:
% - HerEy 38, 114: break in scoring
% - HerEy 39 and 58: page break before
\documentclass[tocdir=../../tmp/B2]{ees}

\begin{document}

\title{[collection – add custom title page]}
\date{2024-08-31}
\license{cc-by-sa-4.0}
\def\MetadataLilypondVersion{2.24.4}
\def\MetadataEESToolsVersion{v2023.10.0}
\repository{edition-esser-skala/eybler-sacred-music}
\version{v2024.08.0}
\checksum{438259b6275ecf0d7a880e64ff81f04eb5b4e8e6}
\def\MetadataQRCode{\relax}
\eesTitlePage

\chapter{Critical Report}

In general, this edition closely follows the respective principal source.
Any changes that were introduced by the editor are indicated
by italic type (lyrics, dynamics and directives), parentheses
(expressive marks and bass figures) or dashes (slurs and ties).
Accidentals are used according to modern conventions.
For further details, consult the Editorial Guidelines
available on the Edition’s webpage.

\section{Abbreviations}

\begin{abbreviations}
\abbr{A}{alto}
\abbr{B}{bass}
\abbr{S}{soprano}
\abbr{T}{tenor}
\abbr{b}{basses}
\abbr{clno}{clarion}
\abbr{cl}{clarinet}
\abbr{cor}{horn}
\abbr{fag}{bassoon}
\abbr{fl}{flute}
\abbr{ob}{oboe}
\abbr{org}{organ}
\abbr{timp}{timpani}
\abbr{trb}{trombone}
\abbr{vla}{viola}
\abbr{vlc}{violoncello}
\abbr{vl}{violin}
\end{abbreviations}

\section{Quem tuus amor ebriat · HerEy 38}

\begin{xltabular}{\linewidth}{@{} >\itshape l X}
genre & offertorium \\
festival & de tempore \\
scoring & [S (solo)], A (solo), S, A, T, B (coro), 2 ob, 2 fag, \newline 2 cor (C), 2 clno (C), 2 trb, timp (C–G), 2 vl, vla, vlc, b, org \\
\end{xltabular}

\begin{sources}

\sourceitem%
  {A1}%
  {A-Ws}%
  {Cod. 707/2}%
  {autograph manuscript (principal source)}%
  {1797}%
  {}%
  {public domain}%
  {}%
  {full score; 28 pages}

\end{sources}

\begin{xltabular}{\linewidth}{ll X}
\toprule
\itshape Bar & \itshape Staff & \itshape Description \\
\midrule \endhead
–   & cor   & cor 1/2 have been added in red ink on the staff of trb~1 (p.~1),
              clno 1/2 (p.~2–4, 6–20), and timp (p.~21–28). \\
–   & trb   & In the \textit{Alleluia}, trb 1 and 2 share a staff with timp
              and clno 1/2, respectively, and are written on p.~21
              (bars 156–161), 25 (bars 201–209), 26 (bars 210–212, 219f),
              and 27 (bars 221–226). Otherwise, they should play unison
              with A and T, as indicated by Eybler’s directive at the bottom
              of p.~21 (“Nb: Wo die Tromboni nicht beſonders ausgeſetzt ſind,
              gehen ſie mit die Singſtimmen.”). \\
– &   S     & Eybler added an alternative part for S~solo on the bottommost
              staff of pages 4–8 and 11–17, and on the staff for S~tutti on
              p.~21–27. \\
1–153 & org & All bass figures have been added by the editor. \\
132   & A   & \A1 contains two autograph candenzas for A~solo
              (on the second staff from the bottom on p.~17):\newline
              \includegraphics{../../front_matter/snippet_38_1.pdf}\newline
              and\newline
              \includegraphics{../../front_matter/snippet_38_2.pdf} \\

\bottomrule
\end{xltabular}

\textlt{\textit{Lyrics}\\Quem tuus amor ebriat,\\ novit quid Jesus sapiat.\\ Quam Felix est quem satiat?\\ Non est ultra quod cupiat.\\[1ex] Jesu decus angelicum,\\ in aure dulce canticum,\\ in ore mel mirificum,\\ in corde nectar coelicum.}

\clearpage
\section{Cantate Domino · HerEy 39}

\begin{xltabular}{\linewidth}{@{} >\itshape l X}
genre & gradual \\
festival & de Tempore \\
scoring & S, A, T, B (coro), 2 ob, 2 clno (C), timp (C–G), 2 vl, vla, vlc, b, org \\
\end{xltabular}

\begin{sources}

\sourceitem%
  {A1}%
  {A-Ws}%
  {566 (3)}%
  {autograph manuscript (principal source)}%
  {1804}%
  {}%
  {public domain}%
  {}%
  {full score; 19 pages}


\sourceitem%
  {D1}%
  {D-NATk}%
  {NA/SP (E-14)}%
  {manuscript not used for this edition}%
  {1813-11-25}%
  {455039863}%
  {public domain}%
  {https://mirador.acdh.oeaw.ac.at/musikarchivspitz/D-NATk_E14/}%
  {15 parts (S, A, T, B, ob 2, clno 1, clno 2, timp, vl 1 (2×) vl 2 (2×), vla, vlne, org)}

\end{sources}


\textlt{\textit{Lyrics}\\Cantate Domino, et benedicite nomini eius.\\ Quoniam magnus Dominus et laudabilis nimis,\\ terribilis est super omnes Deos Dominus.\\ (\bibleverse{Ps}(96/95:2,4))}

\section{Lauda Sion · HerEy 45}

\begin{xltabular}{\linewidth}{@{} >\itshape l X}
genre & sequence \\
festival & in Festo Corporis Christi \\
scoring & S (solo), S, A, T, B (coro), fl, 2 cl (C), 2 cor (C), 2 clno (C), timp (C–G), 2 vl, vla, vlc, b, org \\
\end{xltabular}

\begin{sources}

\sourceitem%
  {B1}%
  {A-Wn}%
  {Mus.Hs.21573}%
  {manuscript copy (principal source)}%
  {no later than 1833}%
  {}%
  {public domain}%
  {https://data.onb.ac.at/rec/AC14266032}%
  {16 parts (S, A (2×), T, B, fl, cl 1, cl 2, cor 1/2, clno 1/2, timp, vl 1, vl 2, vla, vlne, org)}

\end{sources}

\begin{xltabular}{\linewidth}{ll X}
\toprule
\itshape Bar & \itshape Staff & \itshape Description \\
\midrule \endhead
– & – & The setting of the first verse (\textit{Lauda Sion})
        is based on the \textit{Alleluia} movement of HerEy 38.
        It likely has been arranged by a different hand.
        A total of 65 performance dates (as recorded on the envelope of \B1)
        from 1833 to 1906 testify to its popularity;
        the work was not performed only in the years 1860, 1877, 1879f,
        1883–1886, and 1897.
        \B1 also contains settings for three further verses of the hymn
        (\textit{Ecce panis Angelorum}, \textit{In figuris praesignatur},
        and \textit{Bone Pastor, panis vere}), which are likely not by Eybler. \\

\bottomrule
\end{xltabular}



\section{Pater noster · HerEy 52}

\begin{xltabular}{\linewidth}{@{} >\itshape l X}
genre & gradual \\
festival & de tempore \\
scoring & S, A, T, B (coro), 2 cl (\flat B), fag, 2 cor (\flat E), 2 clno (\flat E), 2 vl, vla, vlc, b, org \\
\end{xltabular}

\begin{sources}

\sourceitem%
  {B1}%
  {A-Wn}%
  {Mus.Hs.22390}%
  {manuscript copy (principal source)}%
  {}%
  {600110905}%
  {public domain}%
  {https://data.onb.ac.at/rec/AC14266081}%
  {15 parts (S, A, T, B, cl 1, cl 2, fag, cor 1/2, clno 1/2 vl 1, vl 2, vla, b (2×), org)}

\end{sources}

\begin{xltabular}{\linewidth}{ll X}
\toprule
\itshape Bar & \itshape Staff & \itshape Description \\
\midrule \endhead
–  & –    & Eybler's authorship may be doubted, since the work has
            only survived in a single (mediocre) copy and does not
            appear in his autograph catalogue of works. \\
27 & T    & 2nd \halfNote\ in \B1: f2 \\
37 & B    & 4th \quarterNote\ in \B1: f4 \\
49 & S    & 2nd \halfNote\ in \B1: f″2 \\
49 & B    & 2nd \halfNote\ in \B1: \flat b2 \\
74 & cl 1 & 1st \halfNote\ in \B1: a′4–\crotchetRest \\
74 & vl 2 & 1st \halfNote\ in \B1: g+\flat e′4–\crotchetRest \\
75 & vl 2 & 4th \quarterNote\ in \B1: \flat b+f′4 \\

\bottomrule
\end{xltabular}

\textlt{\textit{Lyrics}\\Pater noster, qui es in coelis,\\ sanctificetur nomen tuum,\\ adveniat regnum tuum,\\ fiat voluntas tua, sicut in coelo, et in terra.\\ Panem nostrum quotidianum da nobis hodie,\\ et dimitte nobis debita nostra,\\ sicut et nos dimittimus debitoribus nostris.\\ Et ne nos inducas in tentationem,\\ sed libera nos a malo.\\ (\bibleverse{Mat}(6:9-13))}

\section{Alma Redemptoris mater · HerEy 57}

\begin{xltabular}{\linewidth}{@{} >\itshape l X}
genre & gradual \\
festival & Mariæ B.V. \\
scoring & S, A, T, B (coro), 2 ob, 2 fag, 2 trb, 2 vl, vla, vlc, b, org \\
\end{xltabular}

\begin{sources}

\sourceitem%
  {A1}%
  {A-Ws}%
  {Cod. 571/4}%
  {autograph manuscript (principal source)}%
  {1815}%
  {}%
  {public domain}%
  {}%
  {full score; 8 pages}


\sourceitem%
  {B1}%
  {A-Wn}%
  {HK.2147/5}%
  {manuscript copy}%
  {1850}%
  {600243147}%
  {public domain}%
  {https://data.onb.ac.at/rec/AC14265952}%
  {full score, 8 pages; lacks trb}


\sourceitem%
  {B2}%
  {A-SPD}%
  {SP (E-3)}%
  {manuscript copy}%
  {1850–1860}%
  {455042253}%
  {public domain}%
  {https://mirador.acdh.oeaw.ac.at/musikarchivspitz/A-SPD_E03/}%
  {11 parts (S, A, T, B, vl 1 (2×), vl 2, vla, vlc, vlne, org)}

\end{sources}


\textlt{\textit{Lyrics}\\Alma Redemptoris Mater,\\ quae pervia coeli porta manes\\ et stella maris:\\ Succurre cadenti,\\ surgere qui curat populo.\\ Tu quae genuisti, natura mirante,\\ tuum sanctum Genitorem,\\ Virgo prius ac posterius,\\ Gabrielis ab ore sumens illud Ave,\\ peccatorum miserere.}

\clearpage
\section{Victimæ paschali laudes · HerEy 58}

\begin{xltabular}{\linewidth}{@{} >\itshape l X}
genre & gradual \\
festival & Resurrectio Domini \\
scoring & S (solo), S, A, T, B (coro), 2 ob, 2 fag, [2 trb], 2 vl, vla, vlc, b, org \\
\end{xltabular}

\begin{sources}

\sourceitem%
  {A1}%
  {A-Ws}%
  {Cod. 571/5}%
  {autograph manuscript (principal source)}%
  {1817-03}%
  {}%
  {public domain}%
  {}%
  {full score; 11 pages}


\sourceitem%
  {B1}%
  {A-Wn}%
  {HK.2147/3}%
  {manuscript copy}%
  {ca. 1850}%
  {600243145}%
  {public domain}%
  {https://data.onb.ac.at/rec/AC14266132}%
  {full score; 14 pages}

\end{sources}

\begin{xltabular}{\linewidth}{ll X}
\toprule
\itshape Bar & \itshape Staff & \itshape Description \\
\midrule \endhead
– & trb & trb are only available in \B1 and thus likely represent a later addition. \\

\bottomrule
\end{xltabular}

\textlt{\textit{Lyrics}\\Victimae paschali laudes immolent Christiani.\\
Agnus redemit oves,\\
Christus innocens Patri reconciliavit peccatores.\\
Mors et vita duello conflixere mirando.\\
Dux vitae mortuus regnat vivus.\\
Dic nobis, Maria: Quid vidisti in via?\\
Sepulchrum Christi viventis et gloriam vidi resurgentis,\\
Angelicos testes sudarium et vestes.\\
Surrexit Christus spes mea,\\
praecedet suos vos in Galilaeam.\\
Scimus Christum surrexisse a mortuis vere.\\
Tu victor rex miserere nobis.
}

\section{Beata gens cuius · HerEy 59}

\begin{xltabular}{\linewidth}{@{} >\itshape l X}
genre & gradual \\
festival & de Sancto Spiritu, in Festo Pentecostes \\
scoring & S, A, T, B (coro), 2 ob, 2 fag, 2 clno (C), timp (C–G), 2 vl, vla, vlc, b, org \\
\end{xltabular}

\begin{sources}

\sourceitem%
  {A1}%
  {A-Ws}%
  {Cod. 571/12}%
  {autograph manuscript (principal source)}%
  {1825}%
  {}%
  {public domain}%
  {}%
  {full score; 19 pages}

\end{sources}


\textlt{\textit{Lyrics}\\Beata gens cuius est Dominus Deus eorum:\\ populus, quem elegit Deus in haereditatem sibi.\\ Verbo Domini coeli firmati sunt,\\ et spiritu oris eius omnis virtus eorum.\\ (\bibleverse{Ps}(33/32:12,6))\\[1ex] Veni Sancte Spiritus,\\ reple tuorum corda fidelium,\\ et tui amoris in eis ignem accende,\\ qui per diversitatem linguarum cunctarum\\ gentes in unitatem fidei congregasti.\\ (Corpus Antiphonalium Officii, no. 5327)}

\section{Salve Regina · HerEy 113}

\begin{xltabular}{\linewidth}{@{} >\itshape l X}
genre & gradual \\
festival & Mariæ B.V. \\
scoring & S, A, T, B (coro), 2 ob, 2 fag, 2 vl, vla, vlc, b, org \\
\end{xltabular}

\begin{sources}

\sourceitem%
  {A1}%
  {A-Ws}%
  {Cod. 707/8}%
  {autograph manuscript (principal source)}%
  {1809-07}%
  {}%
  {public domain}%
  {}%
  {full score; 8 pages}


\sourceitem%
  {B1}%
  {A-Wn}%
  {HK.2147/4}%
  {manuscript copy}%
  {1850}%
  {}%
  {public domain}%
  {https://data.onb.ac.at/rec/AC14266094}%
  {full score; 8 pages}

\end{sources}

\begin{xltabular}{\linewidth}{ll X}
\toprule
\itshape Bar & \itshape Staff & \itshape Description \\
\midrule \endhead
– & – & \A1 contains two versions of the work: Version 1 is for mixed chorus
        with org and vlne (i.e., the bottom three staves in this edition),
        while version 2 is for the same mixed chorus with orchestra
        (i.e., staves 1–5 and 10 in this edition).
        On each of pages 1–7 of \A1, the upper seven staves (of 12 staves
        per page in total) contain parts for S, A, T, B,
        org (right and left hand), and vlne, while the lower four staves
        contain parts for vl 1, vl 2, b/org, and vla.
        Staves 5–7 (i.e., the instruments of version 1) have been
        cancelled with red pencil. The 8th staff from the top is empty,
        except for the first page, where it contains the directive
        “ossia con quest’accompagnamenta /: si copia questo di sotto,
        e gli stromenti da fiato in fine :/”. ob 1/2 and fag 1/2 are written
        separately on page 8, entitled “gli stromenti da fiato”. \\

\bottomrule
\end{xltabular}

\textlt{\textit{Lyrics}\\Salve Regina, mater misericordiae,\\ vita, dulcedo et spes nostra, salve.\\ Ad te clamamus, exules fili Evae.\\ Ad te suspiramus, gementes et flentes\\ in hac lachrymarum valle.\\ Eia ergo, advocata nostra,\\ illos tuos misericordes oculos ad nos converte.\\ Et Jesum, benedictum fructum ventris tui,\\ nobis post hoc exilium ostende.\\ O clemens! O pia! O dulcis virgo Maria.}

\section{Te Deum · HerEy 114}

\begin{xltabular}{\linewidth}{@{} >\itshape l X}
genre & hymn \\
festival & – \\
scoring & S, A, T, B (coro 1), S, A, T, B (coro 2), 2 fl, 2 ob, 2 cl (C), 2 fag, \newline 2 clno (C), 3 trb, timp (C–G), 2 vl, vla, vlc, b, org \\
\end{xltabular}

\begin{sources}

\sourceitem%
  {A1}%
  {A-Ws}%
  {568 (4)}%
  {autograph manuscript (principal source)}%
  {1807}%
  {}%
  {public domain}%
  {}%
  {full score (61 pages); 2 fl, 2 cl, and 3 trb written on 13 separate pages entitled “Stromenti mancanti Per il Te Deum dell’ anno 807”}

\end{sources}

\begin{xltabular}{\linewidth}{ll X}
\toprule
\itshape Bar & \itshape Staff & \itshape Description \\
\midrule \endhead
112 & vl 2 & 1st \eighthNote\ in \A1: a8 \\
353 & S    & upper voice missing in \A1 \\

\bottomrule
\end{xltabular}



\section{Te Deum · HerEy 117}

\begin{xltabular}{\linewidth}{@{} >\itshape l X}
genre & hymn \\
festival & – \\
scoring & S, A, T, B (coro), 2 ob, 2 clno (C), timp (C–G), 2 vl, vla, vlc, b, org \\
\end{xltabular}

\begin{sources}

\sourceitem%
  {A1}%
  {A-Ws}%
  {566 (1)}%
  {autograph manuscript (principal source)}%
  {1804}%
  {}%
  {public domain}%
  {}%
  {full score; 54 pages}

\end{sources}

\begin{xltabular}{\linewidth}{ll X}
\toprule
\itshape Bar & \itshape Staff & \itshape Description \\
\midrule \endhead
161–267 & – & In \A1, vide marks have been added in pencil,
              indicating that the movement may be shortened
              by omitting these bars. \\
257     & B & 1st \quarterNote\ in \A1: e4 \\

\bottomrule
\end{xltabular}



\section{Tristes erant Apostoli · HerEy 123}

\begin{xltabular}{\linewidth}{@{} >\itshape l X}
genre & hymn \\
festival & De tempore paschali \\
scoring & S, A, T, B (coro), b, org \\
\end{xltabular}

\begin{sources}

\sourceitem%
  {A1}%
  {A-Ws}%
  {Cod. 707/4 (2)}%
  {autograph manuscript (principal source)}%
  {}%
  {}%
  {public domain}%
  {}%
  {full score; 1 page}

\end{sources}

\begin{xltabular}{\linewidth}{ll X}
\toprule
\itshape Bar & \itshape Staff & \itshape Description \\
\midrule \endhead
– & – & at end of \A1: “R: Vos elegit Deus in hæreditatem sibi. Alleluja.” \\

\bottomrule
\end{xltabular}

\textlt{\textit{Lyrics}\\Tristes erant Apostoli\\ de Christi acerbo munere,\\ quem morte crudelissima\\ servi necarant impii.\\ (Liber Hymnarius 1983, p. 271)}

\section{Iste confessor · HerEy 124}

\begin{xltabular}{\linewidth}{@{} >\itshape l X}
genre & hymn \\
festival & De Communi Confessorum, in Festo S. Leopoldi Confessoris \\
scoring & S, A, T, B (coro) \\
\end{xltabular}

\begin{sources}

\sourceitem%
  {A1}%
  {A-Ws}%
  {Cod. 707/4 (5)}%
  {autograph manuscript (principal source)}%
  {}%
  {}%
  {public domain}%
  {}%
  {full score; 11 pages}


\sourceitem%
  {B1}%
  {A-Wn}%
  {HK.2485}%
  {manuscript copy}%
  {1835}%
  {}%
  {public domain}%
  {https://data.onb.ac.at/rec/AC14266010}%
  {full score; 4 pages}

\end{sources}


\textlt{\textit{Lyrics}\\Iste confessor Domini colentes,\\ quem pie laudant populi per orbem:\\ Hac die laetus meruit beatas\\ scandere sedes.\\[1ex] Sit salus illi, decus, atque virtus,\\ qui super coeli solio coruscans,\\ totius mundi seriem gubernat\\ Trinus et unus. Amen.\\[1ex] (Liber Hymnarius 1983, p. 466)}


\eesToc{}

\cleardoublepage%
\pagenumbering{arabic}%
\setcounter{page}{1}%
\includepdf[pages=-,link=true,linkname=score]{../../tmp/B2/full_score.pdf}%

\end{document}
