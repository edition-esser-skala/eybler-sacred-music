% created by make_collection.py
% manual adjustments:
% - path of snippet_40.pdf
% - HerEy 40: newline in scoring, C2, E1
% - page break before HerEy 60, 65, 70
% - tocdepth 1
\documentclass[tocdir=../../tmp/B3]{ees}

\setcounter{tocdepth}{1}

\begin{document}

\title{[collection – add custom title page]}
\date{2025-05-01}
\license{cc-by-sa-4.0}
\def\MetadataLilypondVersion{2.24.4}
\def\MetadataEESToolsVersion{v2023.10.0}
\repository{edition-esser-skala/eybler-sacred-music}
\version{v2025.04.0}
\checksum{de9f939e62530be7fe8f8ac5c0385004ba348bf2}
\def\MetadataQRCode{\relax}
\eesTitlePage

\chapter{Critical Report}

In general, this edition closely follows the respective principal source.
Any changes that were introduced by the editor are indicated
by italic type (lyrics, dynamics and directives), parentheses
(expressive marks and bass figures) or dashes (slurs and ties).
Accidentals are used according to modern conventions.
For further details, consult the Editorial Guidelines
available on the Edition’s webpage.

\section{Abbreviations}

\begin{abbreviations}
\abbr{A}{alto}
\abbr{B}{bass}
\abbr{S}{soprano}
\abbr{T}{tenor}
\abbr{b}{basses}
\abbr{clno}{clarion}
\abbr{cl}{clarinet}
\abbr{cor}{horn}
\abbr{fag}{bassoon}
\abbr{ob}{oboe}
\abbr{org}{organ}
\abbr{timp}{timpani}
\abbr{trb}{trombone}
\abbr{vla}{viola}
\abbr{vlc}{violoncello}
\abbr{vl}{violin}
\end{abbreviations}

\section{Omnes de Saba venient · HerEy 40}

\begin{xltabular}{\linewidth}{@{} >\itshape l X}
genre & gradual \\
festival & Epiphania Domini \\
scoring & S (solo), 2 S, 2 A, T, B (coro), 2 ob, 2 fag, 2 cor (C),\newline 2 clno (C), timp (C–G), 2 vl, vla, vlc, b, org \\
\end{xltabular}

\begin{sources}

\sourceitem%
  {A1}%
  {A-Ws}%
  {568 (2)}%
  {autograph manuscript (principal source)}%
  {1807}%
  {}%
  {public domain}%
  {}%
  {full score; 15 pages}


\sourceitem%
  {A2}%
  {A-Wn}%
  {Mus.Hs.16437}%
  {autograph manuscript}%
  {1807}%
  {600110895}%
  {public domain}%
  {https://data.onb.ac.at/rec/AC14266072}%
  {full score; 15 pages}


\sourceitem%
  {C1}%
  {A-Wn}%
  {SA.82.A.22/13,3}%
  {print}%
  {1827}%
  {}%
  {public domain}%
  {https://data.onb.ac.at/rec/AC09228482}%
  {full score; Tobias Haslinger, Wien, plate number 5046}


\sourceitem%
  {C2}%
  {A-Wn}%
  {MS10019-4°}%
  {print}%
  {1827}%
  {}%
  {public domain}%
  {https://data.onb.ac.at/rec/AC09173359}%
  {19 parts (S solo, S rip, A, T, B, ob 1, ob 2, fag 1, fag 2, cor 1, cor 2, clno 1, clno~2, timp, vl 1, vl 2, vla, vlc/b, org); Tobias Haslinger, Wien,\newline plate number 5049}


\sourceitem%
  {E1}%
  {A-Wn}%
  {MS7845-4°/10}%
  {print not used for this edition}%
  {1928}%
  {}%
  {public domain}%
  {https://data.onb.ac.at/rec/AC09173367}%
  {conductor's score; Anton Böhm \& Sohn, Augsburg–Wien,\newline plate number 6847}

\end{sources}

\begin{xltabular}{\linewidth}{ll X}
\toprule
\itshape Bar & \itshape Staff & \itshape Description \\
\midrule \endhead
–   & –     & An in-depth critical report is available in Reinhold
              Kainhofer’s edition of the work (Edition Kainhofer, 2013). \\
–   & cor   & missing in \A1 \\
–   & trb   & These parts have been added by the editor. \\
8   & T     & 2nd \halfNote\ in \A2: c′2 \\
9   & S     & 2nd \halfNote\ in \C2: d″4–g′4 (S solo) and d″4–b′4 (S rip) \\
10  & vl 1  & 3rd \quarterNote\ in \A1: originally e″4, corrected to
              f″4 with pencil; in \A2: f″4; in \C2: e″4 \\
10  & vl 2  & 3rd \quarterNote\ in \A1: originally c″4, corrected to
              d″4 with pencil; in \A2 and \C2: c″4 \\
13–16 & fag & bars in \A2:\newline \includegraphics{../../front_matter/snippet_40.pdf} \\
16  & –     & In \A1, two bars with a short instrumental interlude
              following bar 16 have been cancelled.
              These bars are also absent in all other sources. \\
16  & org   & 3rd \quarterNote\ in \A2: e8.–e16 \\
32  & vl 2  & 1st \quarterNote\ in \A2: f′+c″+f″4 \\
36  & cor 1 & 1st \quarterNote\ in \C2: d″4 \\
37  & A     & 1st \halfNote\ of upper voice in \C2: d″2 \\
106 & S     & 4th \quarterNote\ in \C2 (S solo): f″4 \\

\bottomrule
\end{xltabular}

\textlt{\textit{Lyrics}\\Omnes de Saba venient,\\ aurum et thus deferentes,\\ et laudem Domino annuntiantes.\\ Surge et illuminare Ierusalem,\\ quia gloria Domini super te orta est.\\ (\bibleverse{Isa}(60:6,1))\\[1ex] Vidimus stellam eius in Oriente,\\ et venimus cum muneribus adorare Dominum.\\ (\bibleverse{Mat}(2:2))}

\section{Sperate in Deo · HerEy 41}

\begin{xltabular}{\linewidth}{@{} >\itshape l X}
genre & gradual \\
festival & de Tempore \\
scoring & S, A, T, B (coro), 2 ob, 2 fag, 2 cor (C), 2 trb, 2 vl, vla, vlc, b, org \\
\end{xltabular}

\begin{sources}

\sourceitem%
  {A1}%
  {A-Ws}%
  {Cod. 571/10}%
  {autograph manuscript (principal source)}%
  {1822}%
  {}%
  {public domain}%
  {}%
  {full score; 16 pages}


\sourceitem%
  {A2}%
  {A-Ws}%
  {555 (1)}%
  {autograph manuscript}%
  {1822}%
  {}%
  {public domain}%
  {}%
  {parts for cor 1/2; 4 staves on the last page of the cor parts for the Missa Sancti Mauritii HerEy 4}


\sourceitem%
  {B1}%
  {A-Wn}%
  {HK.2504}%
  {manuscript copy}%
  {1822}%
  {}%
  {public domain}%
  {https://data.onb.ac.at/rec/AC14266100}%
  {38 parts (S (5×), A (5×), T (5×), B (5×), ob 1, ob 2, fag 1/2, cor 1, cor 2, trb 1, trb 2, vl 1 (3×), vl 2 (3×), vla, vlne (2×), org, maestro di capella)}


\sourceitem%
  {C1}%
  {A-Wn}%
  {HK.2144}%
  {print}%
  {1827}%
  {991018076}%
  {public domain}%
  {https://data.onb.ac.at/rec/AC14328859}%
  {full score; Tobias Haslinger, Wien, plate number 5012}


\sourceitem%
  {C2}%
  {A-Wn}%
  {F4.Baden.86}%
  {print}%
  {1827}%
  {654000104}%
  {public domain}%
  {https://data.onb.ac.at/rec/AC14266102}%
  {14 parts (S, A, T, B, ob 1, ob 2, fag 1, fag 2, cor 1/2, vl 1, vl 2, vla, vlne, org); Tobias Haslinger, Wien, plate number 5015}

\end{sources}

\begin{xltabular}{\linewidth}{ll X}
\toprule
\itshape Bar & \itshape Staff & \itshape Description \\
\midrule \endhead
–   & –    & An in-depth critical report is available in Reinhold
             Kainhofer’s edition of the work (Edition Kainhofer, 2011). \\
–   & –    & There exist three versios of the work:
             Initially, it comprised 143 bars (Eybler also wrote down
             this number below the final bar line). For the second version
             (used as template for \B1), Eybler cancelled 17 bars following
             bar 72. (Here, these bars are reproduced as appendix).
             Concomitantly, he crossed out the bar count “143”
             and replaced it by “126”. In the third version
             (used as template for \C1 and \C2), Eybler also cancelled
             bars 80–81 and replaced the bar count by “124”.
             (Here, these bars are indicated by vide marks.)
             The cor parts in \A2 also comprise 124~bars,
             and there are no cancellations. \\
–   & cor  & The title page of \D1 (but no other source) states
             that either two trombones or two horns should play. \\
1   & –    & The addition “con moto” to the tempo indication appears
             in \A1 above the system, \A2, and \C1,
             but not in \A1 below the system, \B1, and \C2.
             Since “con moto” is written in the same ink
             as the bar count “124” in \A1, Eybler may have added it
             when revising the work for the third version. \\
120 & ob 2 & bar in \C1: \crotchetRest–a′2–a′4 \\

\bottomrule
\end{xltabular}

\textlt{\textit{Lyrics}\\Sperate in Deo, omnis congregatio populi,\\ effundite coram illo corda vestra.\\ Deus adiutor noster in aeternum.\\ (\bibleverse{Ps}(62/61:9))}

\section{Domine Deus, omnium creator · HerEy 42}

\begin{xltabular}{\linewidth}{@{} >\itshape l X}
genre & gradual \\
festival & de Tempore \\
scoring & S, A, T, B (coro), 2 ob, 2 fag, 2 clno (C), timp (C–G), 2 vl, vla, vlc, b, org \\
\end{xltabular}

\begin{sources}

\sourceitem%
  {A1}%
  {A-Ws}%
  {Cod. 571/13}%
  {autograph manuscript (principal source)}%
  {1826}%
  {}%
  {public domain}%
  {}%
  {full score; 12 pages}


\sourceitem%
  {C1}%
  {A-Wn}%
  {HK.2154}%
  {print}%
  {1832}%
  {991018065}%
  {public domain}%
  {https://data.onb.ac.at/rec/AC14328838}%
  {full score; Tobias Haslinger, Wien, plate number 5741}


\sourceitem%
  {C2}%
  {A-Wn}%
  {MS69263-4°/7}%
  {print}%
  {1832}%
  {991018065}%
  {public domain}%
  {https://data.onb.ac.at/rec/AC09306796}%
  {16 parts (S, A, T, B, ob 1, ob 2, fag 1, fag 2, clno 1, clno 2, timp, vl 1, vl 2, vla, vlc/b, org); Tobias Haslinger, Wien, plate number 5744}

\end{sources}

\begin{xltabular}{\linewidth}{ll X}
\toprule
\itshape Bar & \itshape Staff & \itshape Description \\
\midrule \endhead
–  & –   & An in-depth critical report is available in Reinhold
           Kainhofer’s edition of the work (Edition Kainhofer, 2011). \\
94 & fag & bass clef missing in \A1 \\

\bottomrule
\end{xltabular}

\textlt{\textit{Lyrics}\\Domine Deus, omnium creator, iustus et misericors,\\ qui solus es bonus, et omnipotens, et aeternus;\\ accipe sacrificium pro universo populo tuo,\\ et custodi partem tuam,\\ et sanctifica partem tuam.\\ (\bibleverse{IMaccabees}(1:24-26))}

\section{Os iusti meditabitur · HerEy 46}

\begin{xltabular}{\linewidth}{@{} >\itshape l X}
genre & gradual \\
festival & de Confessore \\
scoring & S, A, T, B (coro), 2 ob, 2 clno (C), timp (C–G), 2 vl, vla, vlc, b, org \\
\end{xltabular}

\begin{sources}

\sourceitem%
  {A1}%
  {A-Ws}%
  {569 (3)}%
  {autograph manuscript (principal source)}%
  {1805}%
  {}%
  {public domain}%
  {}%
  {full score; 19 pages}

\end{sources}


\textlt{\textit{Lyrics}\\Os iusti meditabitur sapientiam,\\ et lingua eius loquetur iudicium.\\ Lex Dei eius in corde ipsius,\\ et non suplantabuntur gressus eius.\\ (\bibleverse{Ps}(37/36:30-31))}

\section{Omni die dic Mariæ · HerEy 51}

\begin{xltabular}{\linewidth}{@{} >\itshape l X}
genre & gradual \\
festival & de Beatæ Mariæ Virgine \\
scoring & S (solo), 2 cl (\flat B), 2 fag, 2 cor (\flat E), 2 vl, vla, vlc, b, org \\
\end{xltabular}

\begin{sources}

\sourceitem%
  {B1}%
  {A-Wn}%
  {Mus.Hs.9736}%
  {manuscript copy (principal source)}%
  {}%
  {600110954}%
  {public domain}%
  {https://data.onb.ac.at/rec/AC14266075}%
  {11 parts (S solo, cl 1, cl 2, fag 1, fag 2, cor 1, cor 2, vl 1, vl 2, vla, org)}


\sourceitem%
  {B2}%
  {A-Wn}%
  {Mus.Hs.22387}%
  {manuscript copy}%
  {}%
  {600110947}%
  {public domain}%
  {https://data.onb.ac.at/rec/AC14266076}%
  {12 parts (S solo, cl 1, cl 2, fag 1, fag 2, cor 1, cor 2, vl 1, vl 2, vla, vlc, org)}

\end{sources}

\begin{xltabular}{\linewidth}{ll X}
\toprule
\itshape Bar & \itshape Staff & \itshape Description \\
\midrule \endhead
–   & –     & This works is a contrafactum of the aria
              “Liebe! Liebe! Schöpfe ein der Freuden” (2nd act, no. 13)
              from Eybler's opera \textit{Das Zauberschwert} (HerEy 142).
              \B2 represents a shortened version of \B1
              where bars 51–142 have been omitted. \\
–   & org   & All bass figures have been added by the editor. \\
6   & org   & 1st \halfNote\ in \B2: \flat b4.-\quaverRest \\
30  & vl 2  & 3rd \eighthNote\ in \B1: \flat e″8 \\
43  & cl 2  & bar missing in \B1 \\
46  & org   & 1st \eighthNote\ in \B2: \flat E8 \\
48  & org   & 1st \eighthNote\ in \B2: \flat E8 \\
146 & vl 1  & 3rd \quarterNote\ missing in \B1 \\
154 & org   & bar duplicated in \B1 \\
161 & org   & 1st \eighthNote\ in \B2: \flat E8 \\

\bottomrule
\end{xltabular}

\textlt{\textit{Lyrics}\\Omni die dic Mariae\\ mea laudes anima,\\ eius festa, eius gesta\\ cole devotissima.\\[1ex] Contemplare et mirare\\ eius celsitudinis,\\ dic felicem genitricem,\\ dic beatam virginem.\\[1ex] Ipsam cole ut de mole\\ criminum te liberet,\\ hanc appella ne procella\\ vitiorum superet.\\[1ex] Haec persona nobis dona\\ contulit coelestia,\\ hac regina nos divina\\ illustravit gratia.}

\section{Benedicam Dominum · HerEy 55}

\begin{xltabular}{\linewidth}{@{} >\itshape l X}
genre & gradual \\
festival & de tempore \\
scoring & S, A, T, B (coro), 2 ob, 2 cl (\flat B), 2 fag, 2 cor (F), 2 vl, vla, vlc, b, org \\
\end{xltabular}

\begin{sources}

\sourceitem%
  {A1}%
  {A-Ws}%
  {Cod. 571/8}%
  {autograph manuscript (principal source)}%
  {1820-09}%
  {}%
  {public domain}%
  {}%
  {full score (11 pages); cor 1/2 on one separate page}


\sourceitem%
  {C1}%
  {A-Wn}%
  {SA.82.A.22/13,5}%
  {print}%
  {1829}%
  {991018063}%
  {public domain}%
  {https://data.onb.ac.at/rec/AC09228484}%
  {full score; Tobias Haslinger, Wien, plate number 5428}


\sourceitem%
  {C2}%
  {A-Wn}%
  {F4.Baden.89}%
  {print}%
  {1829}%
  {1001234597}%
  {public domain}%
  {https://data.onb.ac.at/rec/AC14265967}%
  {17 parts (S, A, T, B, ob 1, ob 2, cl 1, cl 2, fag 1, fag 2, cor 1, cor 2, vl 1, vl 2, vla, vlc/b, org); Tobias Haslinger, Wien, plate number 5431}

\end{sources}

\begin{xltabular}{\linewidth}{ll X}
\toprule
\itshape Bar & \itshape Staff & \itshape Description \\
\midrule \endhead
– & cor & Eybler cancelled a cor part on the bottommost staff
          in the full score and instead added a new cor part
          on a separate sheet. \\

\bottomrule
\end{xltabular}

\textlt{\textit{Lyrics}\\Benedicam Dominum in omni tempore:\\ semper laus eius in ore meo.\\ In Domino laudabitur anima mea:\\ audiant mansueti et laetentur.\\ (\bibleverse{Ps}(34/33:2–3))}

\clearpage
\section{Peccata dimittis · HerEy 60}

\begin{xltabular}{\linewidth}{@{} >\itshape l X}
genre & gradual \\
festival & de tempore \\
scoring & S, A, T, B (coro), 2 ob, 2 fag, 2 vl, vla, vlc, b, org \\
\end{xltabular}

\begin{sources}

\sourceitem%
  {A1}%
  {A-Ws}%
  {Cod. 571/14}%
  {autograph manuscript (principal source)}%
  {1826}%
  {}%
  {public domain}%
  {}%
  {full score; 12 pages}

\end{sources}

\begin{xltabular}{\linewidth}{ll X}
\toprule
\itshape Bar & \itshape Staff & \itshape Description \\
\midrule \endhead
– & – & In the tempo indication, “con moto” has been added later. \\

\bottomrule
\end{xltabular}

\textlt{\textit{Lyrics}\\Peccata dimittis his, qui invocant te.\\ Ad te Domine faciem meam converto,\\ ad te oculos meos dirigo.\\ (\bibleverse{Tobit}(3:14-15))}

\section{Dominus in Sina in sancto · HerEy 62}

\begin{xltabular}{\linewidth}{@{} >\itshape l X}
genre & gradual \\
festival & Ascensio Domini \\
scoring & S, A, T, B (coro), 2 ob, 2 cl (C), 2 fag, 2 cor (C), 2 vl, vla, vlc, b, org \\
\end{xltabular}

\begin{sources}

\sourceitem%
  {A1}%
  {A-Ws}%
  {Cod. 571/18}%
  {autograph manuscript (principal source)}%
  {1831-11}%
  {}%
  {public domain}%
  {}%
  {full score; 11 pages}

\end{sources}


\textlt{\textit{Lyrics}\\Dominus in Sina in sancto,\\ ascendens in altum,\\ captivam duxit captivitatem.\\ Alleluia.\\ (Cantus Index 008030.1)}

\section{Tu Domine pater noster · HerEy 63}

\begin{xltabular}{\linewidth}{@{} >\itshape l X}
genre & gradual \\
festival & de SS. Nomine Jesu, aut de Tempore \\
scoring & S, A, T, B (coro), 2 ob, 2 fag, 2 cor (C), 2 vl, vla, vlc, b, org \\
\end{xltabular}

\begin{sources}

\sourceitem%
  {A1}%
  {A-Ws}%
  {Cod. 571/20}%
  {autograph manuscript (principal source)}%
  {1836}%
  {}%
  {public domain}%
  {}%
  {full score; 7 pages}

\end{sources}


\textlt{\textit{Lyrics}\\Tu Domine pater noster et redemptor noster,\\ a saeculo nomen tuum.\\ (\bibleverse{Isa}(63:16))}

\section{Benedictus es, Domine · HerEy 64}

\begin{xltabular}{\linewidth}{@{} >\itshape l X}
genre & gradual \\
festival & de Sanctissimæ Trinitate \\
scoring & S, A, T, B (solo), S, A, T, B (coro), 2 cl (\flat B), 2 fag, 2 cor (\flat E), 2 vl, vla, vlc, b, org \\
\end{xltabular}

\begin{sources}

\sourceitem%
  {A1}%
  {A-Ws}%
  {Cod. 571/19}%
  {autograph manuscript (principal source)}%
  {1834}%
  {}%
  {public domain}%
  {}%
  {full score; 6 pages}

\end{sources}


\textlt{\textit{Lyrics}\\Benedictus es, Domine,\\ Deus Patrum nostrorum,\\ et laudabilis et gloriosus in saecula.\\ (\bibleverse{Dan}(3:52))}

\clearpage
\section{Ave Maria · HerEy 65}

\begin{xltabular}{\linewidth}{@{} >\itshape l X}
genre & gradual \\
festival & Mariæ B.V. \\
scoring & S, A, T, B (coro), 2 cl (A), 2 fag, 2 vl, vla, vlc, b, org \\
\end{xltabular}

\begin{sources}

\sourceitem%
  {A1}%
  {A-Ws}%
  {Cod. 571/7}%
  {autograph manuscript (principal source)}%
  {1819}%
  {}%
  {public domain}%
  {}%
  {full score; 11 pages}

\end{sources}


\textlt{\textit{Lyrics}\\Ave Maria, gratia plena,\\ Dominus tecum,\\ benedicta tu in mulieribus,\\ et benedictus fructus ventris tui.\\ (\bibleverse{Luk}(1:28,42))}

\section{Magnificate Dominum mecum · HerEy 67}

\begin{xltabular}{\linewidth}{@{} >\itshape l X}
genre & gradual \\
festival & de tempore \\
scoring & S, A, T, B (coro), 2 ob, clno solo (\flat B), 2 clno (\flat B), timp (\flat B–F), 2 vl, vla, vlc, b, org \\
\end{xltabular}

\begin{sources}

\sourceitem%
  {A1}%
  {A-Ws}%
  {570}%
  {autograph manuscript (principal source)}%
  {1802}%
  {}%
  {public domain}%
  {}%
  {full score; 11 pages}

\end{sources}


\textlt{\textit{Lyrics}\\Magnificate Dominum mecum\\ et exaltemus nomen eius in id ipsum.\\ Exquisivi Dominum et exaudivit me.\\[1ex] (\bibleverse{Ps}(34/33:4-5))}

\clearpage
\section{Iustus ut palma florebit · HerEy 70}

\begin{xltabular}{\linewidth}{@{} >\itshape l X}
genre & gradual \\
festival & de Confessore \\
scoring & S, A, T, B (coro), 2 ob, 2 fag, 2 clno (\flat B), timp (\flat B–F), 2 vl, vla, vlc, b, org \\
\end{xltabular}

\begin{sources}

\sourceitem%
  {A1}%
  {A-Ws}%
  {Cod. 571/22}%
  {autograph manuscript (principal source)}%
  {1807}%
  {}%
  {public domain}%
  {}%
  {full score; 15 pages}

\end{sources}

\begin{xltabular}{\linewidth}{ll X}
\toprule
\itshape Bar & \itshape Staff & \itshape Description \\
\midrule \endhead
– & – & In the original work title (top of fol. 1r) “Offertorium de Confeſsore ad Missam Sti Josephi”, the first word was cancelled by later hand (probably Eybler’s) in pencil and replaced by “Gradual”. The work also appears as gradual no. 2 in Eybler’s autograph catalogue of works. \\

\bottomrule
\end{xltabular}

\textlt{\textit{Lyrics}\\Iustus ut palma florebit,\\ sicut cedrus Libani multiplicabitur\\ in domo Domini.\\ Ad annuntiandum mane misericordiam tuam\\ et veritatem tuam per noctem.\\[1ex] (\bibleverse{Ps}(92/91:13,3))}

\section{Ave Regina cœlorum · HerEy 71}

\begin{xltabular}{\linewidth}{@{} >\itshape l X}
genre & gradual \\
festival & de B. V. M. \\
scoring & S (solo), S, A, T, B (coro), 2 ob, 2 cl (\flat B), 2 fag, 2 vl, vla, vlc, b, org \\
\end{xltabular}

\begin{sources}

\sourceitem%
  {A1}%
  {A-Ws}%
  {Cod. 571/6}%
  {autograph manuscript (principal source)}%
  {1819}%
  {}%
  {public domain}%
  {}%
  {full score; 10 pages}

\end{sources}

\begin{xltabular}{\linewidth}{ll X}
\toprule
\itshape Bar & \itshape Staff & \itshape Description \\
\midrule \endhead
– & – & In \A1, ob and fag are denoted “ad libitum”,
        and cl are labelled “in mancanza degl’oboi”. \\

\bottomrule
\end{xltabular}



\section{Te Deum · HerEy 118}

\begin{xltabular}{\linewidth}{@{} >\itshape l X}
genre & hymn \\
festival & – \\
scoring & S, A, T, B (solo), S, A, T, B (coro), 2 ob, 2 clno (D), timp (D–A), 2 vl, vla, vlc, b, org \\
\end{xltabular}

\begin{sources}

\sourceitem%
  {A1}%
  {A-Ws}%
  {567 (1)}%
  {autograph manuscript (principal source)}%
  {1800-07}%
  {}%
  {public domain}%
  {}%
  {full score; 52 pages}

\end{sources}

\begin{xltabular}{\linewidth}{ll X}
\toprule
\itshape Bar & \itshape Staff & \itshape Description \\
\midrule \endhead
1       & –   & In \A1, the “con fuoco” part of the tempo indication
                has been crossed out with pencil. \\
25      & –   & In \A1, Eybler cancelled three bars following this bar.
                These bars only contain vocal parts. \\
143     & –   & In \A1, Eybler cancelled the original
                tempo indication “Andante”. \\
143–250 & –   & In \A1, vide marks and the directive “bleibt weg”
                (added by later hand in pencil) indicate that this movement
                should be skipped. \\
386f    & vla & In \A1, dashes imply that vla still plays in unison with A.
                However, it is more likely that vla switches to unison
                with b in these bars, as shown here. \\

\bottomrule
\end{xltabular}



\section{Ecce quomodo moritur · HerEy 125}

\begin{xltabular}{\linewidth}{@{} >\itshape l X}
genre & responsorium \\
festival & – \\
scoring & S, A, T, B (coro), [3 trb] \\
\end{xltabular}

\begin{sources}

\sourceitem%
  {A1}%
  {A-Ws}%
  {Cod. 707/3}%
  {autograph manuscript (principal source)}%
  {1816-04-07}%
  {}%
  {public domain}%
  {}%
  {full score; 3 pages}

\end{sources}

\begin{xltabular}{\linewidth}{ll X}
\toprule
\itshape Bar & \itshape Staff & \itshape Description \\
\midrule \endhead
– & trb & Three trombones are implicated by Eybler’s note at the bottom of the last page of \A1: “Nb: Bey den 3 Posaunen wird auch der Text unterlegt.” \\
82–99 & – & In \A1, these bars are indicated by stars in bars 41 and 81, as well as the directive “da capo wird ausgeschrieben”. \\

\bottomrule
\end{xltabular}

\textlt{\textit{Lyrics}\\Ecce quomodo moritur iustus,\\
et nemo percipit corde,\\
et viri iusti tolluntur\\
et nemo considerat.\\
A facie iniquitatis sublatus est iustus:\\
et erit in pace memoria eius.\\
Tamquam agnus coram tondete se obmutuit,\\
et non aperuit os suum:\\
de angustia, et de iudicio sublatus est.\\
(\bibleverse{Isa}(57:1,2); Cantus Index 006605)
}

\section{Cœlestis urbs Jerusalem · HerEy 126}

\begin{xltabular}{\linewidth}{@{} >\itshape l X}
genre & hymn \\
festival & – \\
scoring & S, A, b, org \\
\end{xltabular}

\begin{sources}

\sourceitem%
  {A1}%
  {A-Ws}%
  {Cod. 707/4a (1)}%
  {autograph manuscript (principal source)}%
  {1831}%
  {}%
  {public domain}%
  {}%
  {full score; 2 pages}

\end{sources}


\textlt{\textit{Lyrics}\\Coelestis urbs Jerusalem,\\
beata pacis visio,\\
quae celsa de viventibus\\
saxis ad astra tolleris:\\
Sponsaeque ritu cingeris\\
mille angelorum millibus.\\[1ex]
Decus parenti debitum\\
sit usquequaque Altissimo\\
natoque Patris unico,\\
et inclyto paraclito,\\
cui laus, potestas, gloria\\
aeterna sit per saecula.\\
Amen.\\[1ex]
(Cantus Index a01590)
}

\section{Exultet orbis gaudiis · HerEy 127}

\begin{xltabular}{\linewidth}{@{} >\itshape l X}
genre & hymn \\
festival & – \\
scoring & S, A, b, org \\
\end{xltabular}

\begin{sources}

\sourceitem%
  {A1}%
  {A-Ws}%
  {Cod. 707/4b (1)}%
  {autograph manuscript (principal source)}%
  {}%
  {}%
  {public domain}%
  {}%
  {full score; 1 page}

\end{sources}


\textlt{\textit{Lyrics}\\Exultet orbis gaudiis:\\
Coelum resultet laudibus:\\
Apostolorum gloriam\\
tellus et astra concinunt.\\[1ex]
Patri, simulque Filio,\\
tibique, Sancte Spiritus,\\
sicut fuit, sit iugiter\\
saeclum per omne gloria.\\
Amen.\\[1ex]
(Cantus Index a01579)
}

\section{Tantum ergo · HerEy 128}

\begin{xltabular}{\linewidth}{@{} >\itshape l X}
genre & hymn \\
festival & – \\
scoring & S, A, org \\
\end{xltabular}

\begin{sources}

\sourceitem%
  {A1}%
  {A-Wst}%
  {MHc-1845}%
  {autograph manuscript (principal source)}%
  {}%
  {}%
  {public domain}%
  {https://permalink.obvsg.at/wbr/AC15886650}%
  {full score; 1 page}

\end{sources}




\section{Jesu nostra redemptio · HerEy 130}

\begin{xltabular}{\linewidth}{@{} >\itshape l X}
genre & hymn \\
festival & – \\
scoring & S, A, org \\
\end{xltabular}

\begin{sources}

\sourceitem%
  {A1}%
  {A-Ws}%
  {Cod. 707/4a (3)}%
  {autograph manuscript (principal source)}%
  {}%
  {}%
  {public domain}%
  {}%
  {full score; 1 page}

\end{sources}


\textlt{\textit{Lyrics}\\Jesu nostra redemptio,\\
amor et desiderium,\\
Deus creator omnium,\\
homo in fine temporum.\\
Tu esto nostrum gaudium,\\
qui es futurus praemium,\\
sit nostra in te gloria\\
per cuncta semper saecula.\\
(Cantus Index 008331)
}


\eesToc{}

\cleardoublepage%
\pagenumbering{arabic}%
\setcounter{page}{1}%
\includepdf[pages=-,link=true,linkname=score]{../../tmp/B3/full_score.pdf}%

\end{document}
